0 4118.7998046875 5.1015625
1 4113.3369140625 -0.361328125
2 4113.6982421875 0.0
4 4118.4208984375 4.72265625
6 4125.0927734375 11.39453125
8 4129.82568359375 16.12744140625
